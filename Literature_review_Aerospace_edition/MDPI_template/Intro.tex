\section{Introduction}
Aircraft act as high-altitude emissions vectors, transporting a number of radiatively and chemically active substances across vast regions of the globe. These substances induce a net global warming effect that constitutes 3.5\% of global climate change due to anthropogenic emissions \cite{Lee2021}. Two thirds of this impact result from the climate forcing of non-\ce{CO_2} emissions, primarily through emission of nitrogen oxides (\ce{NO_x}), water vapour (\ce{H_2O}) and particulate matter (PM). These emission species interact with ambient air through chemical and microphysical processing, giving rise to the production and depletion of radiatively active substances that perturb the net energy balance of the atmosphere (e.g. \ce{NO_x}-induced ozone production, condensation trail (contrail) generation through \ce{H_2O} and PM emissions etc.). The degree to which aircraft emissions induce a climatic response varies depending on the state of the background atmosphere (i.e. its chemical composition and meteorology) and the time of day and year on which emissions are released. This means that aviation climate impact is spatio-temporally sensitive, i.e. the same emissions released at different times and/or locations can lead to very different climate effects. 

The dispersion of aircraft emissions occurs over great length and time scales, with emissions entrained in the aircraft exhaust plume which spreads hundreds of kilometres \cite{Kraabol2000a} over its lifetime of up to 12 hours \cite{EPA1992}. The elevated concentrations of emitted chemical species present within the plume result in additional nonlinear chemical (gas-phase and heterogeneous) and microphysical processes which are not accounted for in global chemistry models, due to the inherent assumption of instantaneous dispersion (ID) of emissions. The ID assumption models emissions as homogeneously mixed into the volume of the computational grid cell to which they are released, thus neglecting any subgrid-scale nonlinear processing that may occur throughout the plume lifetime \cite{Paoli2011}. Plume-scale nonlinear effects are also further augmented in high-density airspace regions, where plumes intersect and the emissions contained within them accumulate, leading to chemical saturation. Two key saturation effects have been noted to be of particular significance with respect to aviation climate impact: (1) the saturation of \ce{NO_x} emissions, which leads to decreasing ozone production with increasing \ce{NO_x} concentrations \cite{Jaegle1999}, and (2) the dehydration of water vapour leading to diminished contrail climate impact \cite{Schumann2015}. Therefore, the atmospheric response to non-\ce{CO_2} aircraft emissions is not only sensitive to natural variations in the atmospheric state, but it is also affected by heightened emissions concentrations due to the presence of lingering plumes from previous aircraft.

Optimising flight routes with respect to minimum climate impact (instead of minimum fuel burn) is a concept that has become prevalent in the literature in recent years. This involves re-routing aircraft to avoid particularly climate-sensitive regions of the atmosphere, based on provision of en-route chemical and meteorological information. Nikla{\ss} et al.\ (2019) \cite{Niklass2019} has shown through simulation efforts that it is possible to achieve a 12\% reduction in climate impact at virtually no additional cost to fuel burn, evidencing the practicability of climate-optimised routing. Other studies approach this issue purely from the perspective of superimposing aircraft plumes through formation flight. The aerodynamic benefits of formation flight due to wake energy retrieval are already well established \cite{Bangash2004, Kent2020}, however it has become increasingly evident that formation flight can also be used to exploit climate-beneficial saturation effects due to the superposition of plumes from aircraft involved. In Dahlmann et al.\ (2020) \cite{Dahlmann2020} a preliminary study exploring the potential for climate impact reduction due to saturation effects in formation flight and found that \ce{NO_x} saturation led to a reduction in ozone production efficiency of \textasciitilde5\% and a mutual inhibition of contrail growth, quelling their radiative properties by 20 to 60\%. 

This review article documents the current state of literature on aviation's atmospheric effects and the influence of nonlinear plume-scale processing on the global net climate impact, to bring to light the notion that aviation-induced perturbations to atmospheric chemistry can simply be viewed as another constraint in the climate-optimised routing problem. Section 2 covers the emissions generation process and the methods used to model aircraft performance and emissions. Section 3 explores the dispersion characteristics of aircraft emissions on the subgrid scale, alluding to the various modelling methods developed to investigate the dynamical characteristics of aircraft plumes. In section 4, air traffic management principles and their effect on aviation emissions distribution is considered, on both the local and global scale, followed by section 5 which explores aircraft climate impact, based on the emissions released and the chemical and physical processes that occur on the grid and subgrid-scale of global models. Section 6 looks into the effect of subgrid-scale processes that occur due to emissions accumulation in high-density airspace, and finally section 7 explores potential mitigation efforts to reduce aviation-induced climate change by modifying aircraft operations, such as climate-optimal routing and formation flight.


%This review article aims to bring together literature on these two concepts to prove that the
%This review article documents the current state of literature on aviation's atmospheric effects and the influence of nonlinear plume-scale processing on the global net climate impact, . Section 2 covers the emissions generation process and the methods used to model aircraft performance and emissions. Section 3 explores the dispersion characteristics of aircraft emissions on the subgrid scale, alluding to the various modelling methods developed to investigate the dynamical characteristics of aircraft plumes. In section 4, air traffic management principles and their effect on aviation emissions distribution is considered, on both the local and global scale, followed by section 5 which explores aircraft climate impact, based on the emissions released and the chemical and physical processes that occur on the grid and subgrid-scale of global models. Section 6 looks into the effect of subgrid-scale processes that occur due to emissions accumulation in high-density airspace, and finally section 7 explores potential mitigation efforts to reduce aviation-induced climate change by modifying aircraft operations, such as climate-optimal routing and formation flight.

%. Emissions of \ce{NO_x} indirectly affect the climate through photochemical reactions that generate ozone (\ce{O_3}) and deplete methane (\ce{CH_4}), leading to a net warming effect. In sufficiently cold and humid air, emitted water vapour condenses around PM to form water droplets, which subsequently freeze to form a trail of ice crystals in the aircraft wake. This is The formation of ice crystals in the aircraft wake constitutes what is known as a condensation trail (contrail). Contrails can grow due to uptake of surrounding water vapour and persist for hours, spreading over hundreds of kilometres of the atmosphere, with the possibility of transitioning into cirrus clouds known as contrail cirrus. Persistent contrails and contrail cirrus are known to trap heat more efficiently than they reflect inbound solar radiation, thus exhibiting a net warming effect, thought to be the most significant contribution to aviation-induced climate change \cite{Karcher2018}. 

%The dispersion of aircraft emissions occurs over great length and time scales, with emissions entrained in the aircraft exhaust plume which spreads hundreds of kilometres \cite{Kraabol2000} over its lifetime of up to 12 hours \cite{EPA1992}. The elevated concentrations of emission species present within the plume result in additional nonlinear chemical (gas-phase and heterogeneous) and microphysical processes which are not accounted for in global chemistry models, due to the inherent assumption of instantaneous dispersion (ID) of emissions. The ID assumption models emissions as homogeneously mixed into the volume of the computational grid cell to which they are released, thus neglecting any subgrid-scale nonlinear processing that may occur throughout the plume lifetime \cite{Paoli2011}. The inclusion of plume-scale modelling in the analysis of aviation climate impact captures the initial depletion of ozone in the plume due to the saturation of \ce{NO_x} emissions, followed by the conversion of \ce{NO_x} to nitrogen reservoir species (e.g. \ce{HNO_3}) on the order of 10\%, which are less efficient at producing ozone when propagated to global scales. The lack of accounting of these in-plume processes in the ID approach results in an overestimation of ozone forming potential by up to \textasciitilde20\% \cite{Fritz2020}. Furthermore, heterogeneous reactions that take place on aviation-induced aerosol particles in the plume can lead to the self reaction of a key oxidant known as the hydroperoxy radical (\ce{HO_2}), thus leading to further reduction of ozone production that is overlooked in global modelling efforts \cite{}. Insufficient microphysical modelling of contrails in aircraft exhaust plumes has also been shown to limit accuracy of climate impact calculations in global models \cite{}. The formation and persistence of contrails can be deduced purely from the satisfaction of thermodynamic criteria related to the saturation of ice and water in the surrounding atmosphere. However, determination of the evolution and radiative impact of contrails requires the microphysical analysis of key optical and physical properties, meaning the improper accounting of contrail and aerosol microphysics in global models leads to large uncertainties in the true global atmospheric impact of contrails \cite{}. 

%The ideal solution to the issue of inadequate accounting of subgrid-scale effects in global models would be to obtain high-resolution plume model data for every flight that takes place. In reality, flights take place in the hundreds of thousands every day, so the computational load from such a task would be exceedingly large \cite{}. Instead, plume models and large eddy simulations are used to deduce and calibrate parametrisations of these plume-scale processes, that can be applied to global models to capture the subgrid-scale effects without the need for full plume model implementation. Parametrisations of gas-phase conversions in the plume have been covered extensively in the literature, however the global modelling of heterogeneous chemistry and contrail microphysics is an area that is more difficult to approach, and work is still ongoing to reduce uncertainty in this area \cite{}.

%The degree to which nonlinear chemical and microphysical processing occurs in aircraft plumes varies with respect to temperature, time of emission, latitude, season, turbulence, and the background concentrations of key chemical species, primarily \ce{NO_x} and \ce{O_3} \cite{Kraabol2002}. This means that discrepancies between plume models and global models also deviate with location and time, and any attempts to parametrise plume processes into global models must take this spatio-temporal variability into account. The sensitivity of plume processes to environmental conditions also transfers to scenarios where multiple plumes overlap in high density airspace. The accumulation of emissions due to overlapping plumes in high-density airspace can lead to exceptionally high concentrations of key chemical species, with \ce{NO_x} and cloud condensation nuclei (CCN), such as soot and sulphate aerosol particles, expected to increase the most \cite{Brasseur1998, Schlager1997}. The perturbation to the chemical state of the atmosphere due to other aircraft emissions has been shown to have a considerable effect on the eventual climate impact of further emissions released, with the saturation of \ce{NO_x} leading to further reductions in ozone formation \cite{} and the dehydration of water vapour due to prior contrails leading to a diminishing effect of subsequent contrails released into the same volume of airspace \cite{}. Quantifying aviation's atmospheric impact on subgrid-scales therefore requires determination of the atmospheric state, and the integration of air traffic data, to deduce the perturbation of the atmospheric state due to prior flights through the same region of airspace.

%This review article documents the current state of literature on aviation's atmospheric effects and the influence of nonlinear plume-scale processing on the global net climate impact. Section 2 covers the emissions generation process and the methods used to model aircraft performance and emissions. Section 3 explores the dispersion characteristics of aircraft emissions on the subgrid scale, alluding to the various modelling methods developed to investigate the dynamical characteristics of aircraft plumes. In section 4, air traffic management principles and their effect on aviation emissions distribution is considered, on both the local and global scale, followed by section 5 which explores aircraft climate impact, based on the emissions released and the chemical and physical processes that occur on the grid and subgrid-scale of global models. Section 6 looks into the effect of subgrid-scale processes that occur due to emissions accumulation in high-density airspace, and finally section 7 explores potential mitigation efforts to reduce aviation-induced climate change by modifying aircraft operations, such as climate-optimal routing and formation flight.

%the distribution of air traffic and associated emissions on the global and local scale are . the climate impact of aviation, highlighting the global climate forcing, and how plume-scale processes play a part in this (section 5). 

%Air traffic literature shows that laws of minimum separation do not preclude the overlap of aircraft plumes \cite{}. In fact, in the scenario where two aircraft are pushed to their longitudinal separation minima, the follower aircraft's plume intersects the leader's in a matter of minutes \cite{}.

%Attempts to capture subgrid-scale effects in global models, without full integration of a high resolution plume model 
%To address the issue of inaccuracy in global aviation climate modelling efforts, various theories have been proposed to parametrise the chemical and microphysical effects occurring at the subgrid-scale.

%Parametrisation of gas-phase chemical conversions has been covered extensively in the literature, whereas 


%Just as the magnitude of aviation climate impact at a global scale depends on the time and location with which emissions are released, the degree to which plume-scale chemical and microphysical processing affects the eventual 

%The climate perturbation induced by nonlinear processing in aircraft exhaust plumes is again, entirely spatio-temporally dependent, meaning the

 %\ce{NO_x} to \ce{NO_y} conversion that takes place 

 %has been shown to reduce the net aviation-attributable ozone forming potential by up to \textasciitilde20\% \cite{Fritz2020}. This is because the conversion of \ce{NO_x} to nitrogen reservoir species (\ce{NO_y}) in the plume means that a proportion of \ce{NO_x} emissions are instead propagated to global scales as \ce{NO_y} species which are less efficient at producing ozone. 


%This is because the in-plume conversion of nitrogen oxides (\ce{NO_x}) to nitrogen reservoir species (\ce{NO_y}) is captured in plume models, which leads to  efficient at producing ozone when propagated to global scales \cite{}. Furthermore, the inclusion of heterogeneous chemical processes can further limit the production of ozone and reduce the conversion of \ce{NO_x} to \ce{NO_y} in atmospheric conditions conducive with persistent contrail formation, due to the dehydration and dehoxification of the surrounding atmosphere
%
%Aerosol particles  in the aircraft exhaust plume interact with emitted water vapour, leading to condensation, followed 
%
%Although aviation water vapour contributes a relatively small warming effect compared to other emission species, it does serve as a 
%
%The greenhouse effect due to aviation water vapour is a relatively small contribution to aviation-induced climate
%
%The climate contribution of aviation water vapour emissions is relatively small does have a small direct climate forcing due to the induced greenhouse effect, however its radiative impact predominantly derives from the formation of condensation trals 

%(1) the influence of nitrogen oxide (\ce{NO_x}) emissions on atmospheric ozone and methane concentrations and (2) the formation of ice clouds known as condensation trails (contrails) due to condensation of water vapour on  and heterogeneous freezing of water vapour on aerosol particles in the aircraft wake, and x


%The extent to which the ID assumption leads to loss of accuracy in global aviation climate modelling is 
%
%To address this lack of
%
%In the gas-phase, high levels of nitrogen oxides (\ce{NO_x}) in the plume locally deplete ozone (\ce{O_3}) levels, such that the eventual recovery in ozone production 
%
%the presence of the  nonlinear chemistry effects include more efficient conversion of nitrogen oxides (\ce{NO_x}) to nitrogen reservoir species in the plume, resulting in a net reduction in the ozone forming potential of \ce{NO_x} emissions compared to the equivalent released in an ID scenario.
%
%gas-phase chemical conversions of NOx to NOy in the plume which , thus
%
%The chemistry that 
%
%These nonlinear effects have a potentially significant influence on net climate impact, yet are not accounted for in global chemistry models due to the inherent assumption of instantaneous dispersion (ID) of emissions into the volume of the computational grid cell in which they are released into.  that neglects the sub-grid scale nonlinear processes that occur due to the enhanced concentrations of chemical species present in the exhaust plume throughout its lifetime \cite{Paoli2011}. 
% In high-density airspace, such as along the North Atlantic Flight Corridor, aircraft often fly in close proximity, giving rise to the superposition of their plumes and the accumulation of emissions contained within them \cite{Schlager1997}. 
%
%
%It has been shown in \cite{} that the overlap of four consecutive plumes can influence the nonlinear chemistry and microphysics significantly, leading to saturation effects that can potentially provide a net climate impact reduction.
%
%Following expulsion into the free atmosphere, aircraft exhaust species become entrained in the aircraft wake, forming a plume of elevated chemical concentrations which persist for 2 to 12 hours \cite{}, before fully dispersing into ambient air. The build-up of non-CO2 emissions in the plume give rise to a number of nonlinear chemical and microphysical effects, which influence the ensuing atmospheric response by altering the net production rates of radiatively active gases and affecting contrail formation and persistence. It is often the case however, that in large-scale atmospheric models, emissions are instantaneously diluted into the volume of the smallest resolved grid cell, with dimensions according to the model’s spatial resolution. The instantaneous dilution (ID) approach neglects the plume-scale processes, thus leading to discrepancies in the calculated climate response such as the overestimation of O3 production, CH4 and CO destruction, and the increased rate of NOx conversion to nitrogen reservoir species \ref{}. Furthermore, it is stated in \ref{} that the formation of ice in aircraft exhaust plumes may result in additional heterogeneous chemical reactions, that are not captured in global atmospheric models. 
%
%% Introduce each section and explain whats in them and how whole review pieces together and the purpose of the review