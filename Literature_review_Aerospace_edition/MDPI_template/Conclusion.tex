\section{Conclusions}
This review paper delivers a holistic overview of literature pertaining to the spatio-temporal climate sensitivity of the atmosphere to reactive non carbon-based species. This includes literature from a range of research areas: aircraft emissions, plume dispersion and dynamics, the distribution of air traffic and corresponding emissions, the climate impact of aviation on both a global and a local scale, and lastly the mitigation potential for both \ce{CO_2} and non-\ce{CO_2} climate impact. 

The generation of aircraft emissions firstly requires consideration and modelling of aircraft performance, to determine fuel consumed throughout flight. This is then followed by the modelling of emission formation mechanisms, to determine the emission indices of each constituent chemical substance that is generated during jet fuel combustion. Emissions modelling determines the mass of emissions generated by aircraft, but does not provide essential information about the perturbations to chemical concentrations in the surrounding atmosphere. To address this, dispersion modelling is required to determine the evolution of emissions concentrations over time, through consideration of the dynamical processes that ensue once emitted from the aircraft exhaust. 

Characteristic global air traffic distribution patterns are observed from a range of aircraft emissions datasets, indicating distinct spatial and temporal trends in air traffic flows. Inhomogeneities in the distribution of air traffic with respect to both time and location, directly affect the global atmospheric response to non-\ce{CO_2} emissions, such as water vapour and \ce{NO_x}. These distribution trends are largely driven by the key principles of air traffic management and typical aviation demand patterns. Minimum separation laws determine the theoretical upper limit on airspace density for a given air traffic sector, however it is airspace sector capacity that severely limits how many aircraft can be present in the same region at any one time. Therefore, if the aim is to maximise airspace density so that aircraft have more flexibility in route selection, the industry must push for airspace modernisation through more automation, integration and collaboration of air traffic flows. Air traffic demand also has a significant role to play in distribution trends, with flight densities skewed more towards daytime, summer months, typical aircraft cruising altitudes, and above more densely-populated geographical locations and their connecting regions.

The generation, dispersion and distribution of aircraft emissions are then observed in the context of aviation climate impact and the chemical and microphysical response of the atmosphere. Globally, the radiative forcing contributions to aviation climate impact are discussed, as well as the typical metrics and modelling methods used to analyse global aviation climate impact. Nonlinear plume-scale effects are then observed, with both gas-phase photochemical, as well as heterogeneous chemical processes discussed. The typical upper tropospheric reaction mechanisms for gas-phase photochemistry are delineated, as these provide an explanation for the production and depletion of ozone and methane due to an input of \ce{NO_x} emissions.

Finally, mitigation strategy is discussed from the perspective of both the conventional net zero decarbonisation approach, and an alternative approach which incorporates non-\ce{CO_2} mitigation through operational measures such as climate-optimal aircraft routing and formation flight. 

KEY FINDINGS AND LIMITATIONS!?

% Emissions generation and the concept of primary and secondary combustion products. How these affect the atmosphere in different ways.
% Production mechanism of each of the key species
% Emissions modelling methods - first modelling fuel consumption, then estimating aircraft emissions using a variety of models at varying fidelities.
% Emissions inventories and how they are used to model emissions on a large scale (regional, global etc.).

% Dispersion
% Regimes of aircraft plume dispersion - emissions modelling determines the mass of emissions generated through the combustion process. Dispersion modelling is required to determine the evolution of emissions concentrations over time, through consideration of the  dynamical processes that ensue following release from the aircraft exhaust.
% Various modelling approaches have been developed to capture the dispersion of emissions. These range from low complexity empirical models like the Schumann dilution model, to the intermediate Single and Multi-layered Plume models, to high-resolution large eddy simulations.

% Air traffic
% The key principles of air traffic management and aviation demand patterns drive the distribution of air traffic both globally and locally. Minimum separation laws determine the theoretical upper limit on airspace density for a given air traffic sector, however it is airspace sector capacity that severely limits how many aircraft can be present in the same region at any one time. Therefore, if the aim is to maximise airspace density so that aircraft have more flexibility in route selection, the industry must push for more airspace modernisation through more automation, integration and collaboration of air traffic flows. Characteristic global air traffic distribution patterns are observed from a range of aircraft emissions datasets, indicating distinct spatial and temporal trends in the data. Inhomogeneities in the distribution of air traffic with respect to both time and location, directly affect the 

leads to an inhomogeneous dwhich affects the resulting atmospheric response of reactive non-\ce{CO_2} species





% Maximum ozone depletion in c
% More neutral paper than policy paper



%\section{Plume-scale processes}
%
%Following expulsion into the free atmosphere, exhaust  species become entrained in the aircraft wake, forming a plume of elevated chemical concentrations which persist for 2 to 12 hours \ref{}, before fully dispersing into ambient air. The build-up of non-CO2 emissions in the plume give rise to a number of nonlinear chemical and microphysical effects, which influence the ensuing atmospheric response by altering the net production rates of radiatively active gases and affecting contrail formation and persistence. It is often the case however, that in large-scale atmospheric models, emissions are instantaneously diluted into the volume of the smallest resolved grid cell, with dimensions according to the model’s spatial resolution. The instantaneous dilution (ID) approach neglects the plume-scale processes, thus leading to discrepancies in the calculated climate response such as the overestimation of O3 production, CH4 and CO destruction, and the increased rate of NOx conversion to nitrogen reservoir species \ref{}. Furthermore, it is stated in \ref{} that the formation of ice in aircraft exhaust plumes may result in additional heterogeneous chemical reactions, that are not captured in global atmospheric models. 
%
%\subsection{Plume superposition}
%
%\subsection{Atmospheric response to plume superposition}
%In \ref{}, the notion of plume superposition and the extent to which it influences the atmospheric response was explored, by comparing results from an expanding plume model against an atmospheric model which assumes instant dilution. The analysis was carried out firstly on a single aircraft at cruise altitude, then on four aircraft flying in the same direction each separated by 1 hr. Observing the single aircraft’s plume 10 hrs after emission, it was found that the instantaneous dilution approach overestimated O3 production by 33\%, CH4 destruction by 30\% and CO destruction by 32\%. When observing the response due to the four overlapping plumes however, the overestimation of O3 production increases to 77\%, CH4 destruction to 68\% and CO destruction to 74\%. The marked difference in measured response for four overlapping plumes compared to a single plume serves as evidence that emissions saturation in superimposed plumes can significantly alter the chemical response of the atmosphere. However, further work needs to be done to clarify the direction and magnitude of the net change in climate impact resulting from these changes to the chemical composition. Moreover, the sensitivity of the chemical and climate response to plume superposition needs to be tested for a range of aircraft transit frequencies and ambient atmospheric conditions, so as to build up a visual representation over the phase space of the atmosphere.
%In addition to chemical changes in the atmosphere, plume overlap can also influence the formation and persistence of aircraft contrails and the subsequent generation of aviation-induced cirrus clouds. \ref{} explored the effect of recurrent water vapour emission on contrail formation and persistence at typical aircraft cruising altitudes. It was discovered that continual contrail generation actually leads to dehydration of the surrounding atmosphere and a diminishing radiative forcing effect of ~15\% in the affected areas. This outcome provides further impetus to explore the potential effects of plume superposition on the global warming induced by aviation.
